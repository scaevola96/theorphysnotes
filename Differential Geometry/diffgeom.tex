\documentclass[a4paper,12pt]{article}
%%% Работа с русским языком
\usepackage{cmap}					% поиск в PDF
\usepackage[T2A]{fontenc}			% кодировка
\usepackage[utf8]{inputenc}			% кодировка исходного текста
\usepackage[english,russian]{babel}	% локализация и переносы
%%
\usepackage{cmap} 
\usepackage{ gensymb }
\usepackage[unicode]{hyperref}
\usepackage{ textcomp }
\usepackage{datetime}
\usepackage{physics}
\usepackage{cancel}
\usepackage{mathtools}
\usepackage[margin=0.7in]{geometry}
\usepackage{fancyhdr}
\pagestyle{fancy}
%%
%%% Дополнительная работа с математикой
\usepackage{amsfonts,amssymb,amsthm,mathtools} % AMS
\usepackage{amsmath}
\usepackage{icomma} % "Умная" запятая: $0,2$ --- число, $0, 2$ --- перечисление

%% Шрифты
\usepackage{euscript}	 % Шрифт Евклид
\usepackage{mathrsfs} % Красивый матшрифт

\newcommand*{\hm}[1]{#1\nobreak\discretionary{}
	{\hbox{$\mathsurround=0pt #1$}}{}}

\usepackage{graphicx}  % Для вставки рисунков
\graphicspath{{images/}{images2/}}  % папки с картинками
\setlength\fboxsep{3pt} % Отступ рамки \fbox{} от рисунка
\setlength\fboxrule{1pt} % Толщина линий рамки \fbox{}
\usepackage{wrapfig} % Обтекание рисунков и таблиц текстом
\title{Differential Geometry}
\date{\today}
\author{Sergey Barseghyan}
%% Теоремы Определения Леммы
\newtheorem{exercise}{Упражнение}[section]
\newtheorem*{solution}{Решение}
\newtheorem{definition}{Определение}
\newtheorem{theorem}{Теорема}
\newtheorem{example}{Пример}
\newtheorem{statement}{Утверждение}



%% Коммутатор




\begin{document} % конец преамбулы, начало документа
	
	\maketitle
	\section{Семинар 1}
\begin{exercise}
	Покажите, что унитарные матрицы, как и эрмитовы, диагонализуемы. Указание: покажите, что эрмитова и анти-эрмитова
	часть унитарного оператора диагонализуемы совместно
\end{exercise}	

\begin{solution}
	$$
	U^{\dagger} U=U U^{\dagger}=I
	$$
	
	Матрица $A$ диагонализируема если
	$$
	\exists P: \hspace*{5pt} P^{-1} A P=\left(\begin{array}{cccc}{\lambda_{1}} \\ {} & {\lambda_{2}} \\ {} & {} & {\ddots} \\ {} & {} & {} & {\lambda_{n}}\end{array}\right)
	$$
	
	$$
	U = A + B = \underbrace{\frac{U + U^{\dagger}}{2}}_{\text{эрмитова}} \hspace{5pt} +\underbrace{\frac{U - U^{\dagger}}{2}}_{\text{антиэрмитова }}
	$$
	
	Две матрицы $A \text{ и } B$ совместно диагонализируемы $\leftrightarrow$ $\commutator{A}{B} = 0 $
	
	$$
	\begin{array}{cccc}
	\commutator{A}{B}=\frac{1}{4}\left(\left(U + U^{\dagger}\right)\left(U - U^{\dagger}\right) + \left(U - U^{\dagger}\right)\left(U + U^{\dagger}\right)\right)\\
	=\frac{1}{4}\left(U U + U^{\dagger} U - U U^{\dagger} - U^{\dagger} U^{\dagger}\right) - \frac{1}{4}\left(U U - U^{\dagger} U + U U^{\dagger} - U^{\dagger} U^{\dagger}\right) = 0
	\end{array}
	$$
\end{solution}	
	
\end{document}