\section{Семинар 2}
\begin{exercise}
	Вычислите среднее значение спина $\langle\boldsymbol{S}\rangle$ и его дисперсию $\left\langle(\boldsymbol{S}-\langle\boldsymbol{S}\rangle)^{2}\right\rangle$ для чистого $\ket{\chi}= (\ket{\uparrow}+\ket{\downarrow}) / \sqrt{2}$ 2 и смешанного $\hat{\rho}=\left(\hat{\mathbb{P}}_{\uparrow}+\hat{\mathbb{P}}_{\downarrow}\right) / 2$  состояний спина $1/ 2$. \textit{Комментарий:}  первая величина — это вектор, а вторая
	— это скаляр, длина вектора. \textit{Указание:} Для частицы со спином $1 / 2$ 2 (например, электрон) \textit{оператор спина} а (собственного
	момента)  равен $\hat{\boldsymbol{S}}=(\hbar / 2) \hat{\boldsymbol{\sigma}}$ (то есть $\hat{S}_{x}=(\hbar / 2) \hat{\sigma}_{x}$ и т. д.).
	
\end{exercise}	

\begin{solution}
	$\langle\boldsymbol{S}\rangle$ = $\bra{\chi}\hat{\boldsymbol{S}}\ket{\chi} = $
	$\displaystyle \frac{\hbar}{2} \cdot \frac{1}{2} \left(\begin{array}{l}{1} \hspace{7pt} {1}\end{array}\right)
	 \left(\begin{array}{c}{\hat{\sigma}_{x}} \\ {\hat{\sigma}_{y}} \\ {\hat{\sigma}_{z}}\end{array}\right) \left(\begin{array}{l}{1} \\ {1}\end{array}\right)= $ 
	 $
	 	\frac{\hbar}{2} \left(\begin{array}{l}{1} \\ {0} \\ {0}\end{array}\right)
	 $
	 $$
	 \left\langle(\boldsymbol{S}-\langle\boldsymbol{S}\rangle)^{2}\right\rangle= \bra{\chi}( 
	 \boldsymbol{S}-\langle\boldsymbol{S}\rangle)^{2}  \ket{\chi} =\frac{\hbar^2}{4\cdot 2} \left(\begin{array}{l}{1} \hspace{7pt} {1}\end{array}\right)
	 \left(\left(\hat{\sigma}_x -1\right)^2+ \hat{\sigma}_{y}^2 +\hat{\sigma}_{z}^2\right)
	 \left(\begin{array}{l}{1} \\ {1}\end{array}\right)
	 $$
	 
	 $$
	 \frac{\hbar^2}{4\cdot 2} \left(\begin{array}{l}{1} \hspace{7pt} {1}\end{array}\right)
	 \left(3\mathbb{I} -2\hat{\sigma}_x +1\right)
	 \left(\begin{array}{l}{1} \\ {1}\end{array}\right) = \frac{\hbar^2}{2}
	 $$
	 
	 
	 
	 $$
	 \hat{\rho}=\frac{1}{2}\left(\begin{array}{ll}{1} & {1} \\ {1} & {1}\end{array}\right)
	 $$
	 
	 $$
	 \operatorname{Tr}\hat{\rho}\hat{\boldsymbol{S}} = \frac{\hbar}{2} \operatorname{Tr}\frac{1}{2}\left(\begin{array}{ll}{1} & {1} \\ {1} & {1}\end{array}\right)
	 \left(\begin{array}{c}{\hat{\sigma}_{x}} \\ {\hat{\sigma}_{y}} \\ {\hat{\sigma}_{z}}\end{array}\right)=\frac{\hbar}{2}
	  \left(\begin{array}{c}\operatorname{Tr}\left(\begin{array}{cc}\frac{1}{2} & \frac{1}{2} \\
	 \frac{1}{2} & \frac{1}{2} \\ \end{array}\right) \\ \operatorname{Tr} \left(\begin{array}{cc}
	\frac{i}{2} & -\frac{i}{2} \\ \frac{i}{2} & -\frac{i}{2} \\ \end{array} \right) \\
	\operatorname{Tr} \left(\begin{array}{cc} \frac{1}{2} & -\frac{1}{2} \\ \frac{1}{2} & -\frac{1}{2} \\	\end{array} \right) \\ \end{array} \right) = \frac{\hbar}{2} \left(\begin{array}{l}{1} \\ {0} \\ {0}\end{array}\right)
	  $$
	  
	  
	  $$
	  \operatorname{Tr} \hat{\rho} \left(\boldsymbol{S}-\langle\boldsymbol{S}\rangle\right)^{2} = \operatorname{Tr} \hat{\rho} \left(\left(\hat{\sigma}_{x}-1\right)^{2}+\hat{\sigma}_{y}^{2}+\hat{\sigma}_{z}^{2}\right) = \operatorname{Tr} \frac{1}{2}\left(\begin{array}{ll}{1} & {1} \\ {1} & {1}\end{array}\right) \left(\begin{array}{cc} 2 & -1 \\ -1 & 2 \\\end{array}\right) = \frac{\hbar^2}{2}
	  $$
	 
	 
\end{solution}


\begin{problem}{Блоховское представление двухуровневой системы}
	\begin{enumerate}
		\item 
		Покажите, что матрицу плотности произвольной двухуровневой системы самого общего вида можно разложить по
		матрицам Паули в следующем виде:
		$$
		\hat{\rho}=\frac{1}{2}(\hat{\mathbb{I}}+\hat{\boldsymbol{\sigma}} \cdot \boldsymbol{n})
		$$
		
		\item 
		При каком условии на $ \boldsymbol{n} $, эта матрица плотности описывает чистое состояние?
		
		\item 
		Вычислите средние значения $\left\langle\hat{\sigma}_{x, y, z}\right\rangle$  по состоянию, описываемому такой матрицей плотности.
	\end{enumerate}
\end{problem}

\begin{solution}
	\begin{enumerate}
		\item
		$$
		\hat{\rho}=\sum_{\alpha} p_{\alpha}\ket{\alpha}\bra{\alpha}
		$$
		
		Так как $ \hat{\rho} = \hat{\rho}^\dagger $  тогда $ \hat{\rho} $ можно раложить по базису 
		$
		\hat{\rho} = \hat{\boldsymbol{\sigma}}\boldsymbol{n} + n_0 \sigma^0
		\qquad
		\operatorname{Tr} \hat{\rho}=1
		$
		
		$
		\operatorname{Tr} \hat{\sigma}^{i}=0 \qquad i \in \left\lbrace 1,2,3\right\rbrace
		$
		
		$
		\operatorname{Tr} \hat{\sigma}^{0}=2 \Longrightarrow n_{0}=\frac{1}{2}
		$
		
		$
		\hat{\rho}=\frac{1}{2}\left(\hat{\hat{\sigma}}^{0}+n^{i} \hat{\sigma}_{i}\right)
		$
		
		$
		\bra{\psi}\hat{\rho}\ket{\psi}\geq 0
		\quad
		\hat{\rho} = 
		\frac{1}{2}\left(\begin{array}{ll}{1+n_{z}} & {n_{x} -i n_{y}} \\ n_{x}+{i n_{y}} & {1-n_{z}}\end{array}\right)
		$
		
		
		$
		\left\lbrace\begin{array}{l}{\left|n_{z}\right| \leq 1} \\ {1-n_{z}^{2}-n_{x}^{2}-n_y^{2} \geq 0}\end{array}\right.
		\Longrightarrow
		\vec{n} \in \bar{B}_{1}(0)
		$
		
		\item 
		
		$
		\hat{\rho}^{2}=\hat{\rho}
		$
		
		$
		\hat{\rho}^{2}=\frac{1}{4}\left(\hat{\sigma}_{0}+n_i \hat{\sigma}_{i}\right)^{2} =\frac{1}{4} \left(
		\left(\overline{n} \overline{\sigma}\right)^{2}+2\left(n_{0} \sigma_0\right)(\overline{n} \overline{\sigma})+\left(n_{0} \sigma_0 \right)^{2}
		\right) = 
		\frac{1}{4} \hat{\sigma}_{0}+\frac{1}{2} n^{i} \hat{\sigma}_{i}+\frac{1}{4}(\vec{n})^{2} \hat{\sigma}_{0} =
		$
		
		$
		=\frac{1}{2}\left(\frac{1+|\vec{n}^2|}{2}\sigma_{0} +n^i \sigma_{i}\right)
		$
		
		$	
		\frac{1+|\vec{n}^2|}{2}=1 \Longrightarrow 	|\vec{n}|^{2}=1 \Longleftrightarrow 
		\vec{n} \in \mathbb{S}^{2}
		$
		
		\item 
		$
		\langle \hat{\sigma_{i}}\rangle = \operatorname{Tr}\left( \frac{1}{2}\left( \hat{\sigma_{0}}+ n^k \hat{\sigma}_k\right) \hat{\sigma}_{i} \right)
		$
		
		
		$
		\operatorname{Tr}\left(\frac{1}{2}\left(\hat{\sigma}_{i}+n^{k}\left(\delta_{ki} \hat{\sigma_{0}}+\varepsilon_{kin} \hat{\sigma}_{n}\right)\right)\right) = 
		n_{i}
		$
		
		$
		\langle\hat{\sigma}\rangle=\vec{n}
		$
		
	\end{enumerate}
\end{solution}	


\begin{problem}{Термодинамика двухуровневой системы}
	Двухуровневая система описывается гамильтонианом $
	\hat{H}=-\boldsymbol{h} \cdot \hat{\boldsymbol{\sigma}}
	$, где $
	\boldsymbol{h}=\left(h_{x}, h_{y}, h_{z}\right)
	$ , и находится при температуре $ T $. Вычислите средние значения $\left\langle\hat{\sigma}_{x, y, z}\right\rangle$.
\end{problem}	

\begin{solution}
	$\displaystyle 
	\hat{\rho}=\frac{1}{Z} e^{-\beta \hat{H}} = \frac{1}{Z} e^{-\frac{1}{T} \hat{H}} = \frac{1}{Z} e^{\frac{1}{T} h_x \hat{\sigma}_{x}+h_y \hat{\sigma}_{z}+h_z \hat{\sigma}_{z}}$
	
	
	$ \displaystyle 
	Z = \operatorname{Tr}	= \frac{1}{Z} e^{\frac{|h|}{T} \left(\frac{h_x \hat{\sigma}_{x}}{|h|}+\frac{h_y \hat{\sigma}_{y}}{|h|}+\frac{h_z \hat{\sigma}_{z}}{|h|}\right)} 
	$
	
	$ \displaystyle 
	\operatorname{Tr}\left(\cos \left(-i \frac{|\vec{h}|}{T}\right)+\sin \left(-\frac{-i |\vec{h}|}{T}\right) \cdot \frac{h_\alpha \hat{\sigma}^{\alpha}}{|\vec{h}|}\right)=
	\operatorname{Tr}\left(\operatorname{ch } \left(\frac{|\vec{h}|}{T}\right)+\operatorname{sh}\left(\frac{|\vec{h}|}{T}\right) \cdot \frac{h_{\alpha} \sigma^{\alpha}}{|\vec{h}|}\right) = 2 \operatorname{ch} \frac{|\vec{h}|}{T}
	$
	
	$ \displaystyle
	\hat{\rho}=\frac{e^{\frac{1}{T}\left(h_{\alpha} \hat{\sigma}^{\alpha}\right)}}{2 \operatorname{ch} \frac{|\vec{h}|}{T}} = \frac{1}{2}\frac{\operatorname{ch}\frac{|\vec{h}|}{T} \cdot \mathbb{I} + \operatorname{sh}\frac{|\vec{h}|}{T} \frac{h_{\alpha} \hat{\sigma}^{\alpha}}{|\vec{h}|}}{\operatorname{ch}\frac{|\vec{h}|}{T}} = \frac{1}{2}\left(\mathbb{I} + \operatorname{th}\frac{|\vec{h}|}{T}\frac{h_{\alpha} \hat{\sigma}^{\alpha}}{|\vec{h}|}\right)
	$
	
	
	$
	\left\langle \hat{\sigma}_{x}\right\rangle =\operatorname{Tr}\left(\hat{\rho} \hat{\sigma}_{x}\right)= \frac{1}{2} \operatorname{Tr}\left(\left(\mathbb{I}+\operatorname{th} \frac{|\vec{h}|}{T} \frac{h_{\alpha} \hat{\sigma}^{\alpha}}{|\vec{h}|}\right) \hat{\sigma}_{x}\right) = \frac{h_x}{h} \operatorname{th}\left(\frac{|\vec{h}|}{T}\right)
	$
	\\
	Аналогично
	\\
	
	$ 	\left\langle \hat{\sigma}_{y}\right\rangle = \frac{h_y}{h} \operatorname{th}\left(\frac{|\vec{h}|}{T}\right) $
	$ 	\left\langle \hat{\sigma}_{z}\right\rangle = \frac{h_z}{h} \operatorname{th}\left(\frac{|\vec{h}|}{T}\right) $
	
\end{solution}	


\begin{problem}
	Рассмотрите двухуровневую систему, описываемую следующим гамильтонианом
	$$\hat{H}=\left(\begin{array}{cc}{E_{0}} & {-\Delta} \\ {-\Delta} & {E_{0}}\end{array}\right)$$
	В начальный момент система приготовлена в состоянии $ \ket{\psi(0)} = \ket{\uparrow} $ . Если бы мы позволилит системе эволюционировать самой по себе, то она совершала бы осцилляции Раби; в частности, через время $T=\frac{\pi \hbar}{2 \Delta}$ мы бы обнаружили её в состоятнии $ \ket{\downarrow} $, c вероятностью $P_{\downarrow}(T)=1$. Однако теперь вместо этого через каждый промежуток времени $\tau \ll T$ мы проводим измерение наблюдаемой $ \hat{\sigma}_{z} $. Определите вероятность $P_{\downarrow}(T)$ в таком случае.
\end{problem}

\begin{solution}
$	\ket{\psi(t)} = e^{\frac{-i E_0 t}{\hbar}} \left(\cos \frac{\Delta t}{\hbar}\cdot\ket{\uparrow}+i \sin \frac{\Delta t}{\hbar}\cdot\ket{\downarrow}\right)$

$$
\hat{\rho}=\left(\begin{array}{c}{\cos ^{2}\left(\frac{\Delta t}{\hbar}\right) \quad-i \sin \left(\frac{\Delta t}{\hbar}\right) \cos \left(\frac{\Delta t}{\hbar}\right)} \\ {i \sin \left(\frac{\Delta t}{\hbar}\right) \cos \left(\frac{\Delta t}{\hbar}\right) \quad \sin ^{2}\left(\frac{\Delta t}{\hbar}\right)}\end{array}\right)
$$
После измерения

$$\hat{\rho} \rightarrow \hat{\rho}^{\prime}=\left(\begin{array}{cc}{\cos ^{2}\left(\frac{\Delta t}{\hbar}\right)} & {0} \\ {0} & {\sin ^{2}\left(\frac{\Delta t}{\hbar}\right)}\end{array}\right)$$

$$
\hat{\rho}(t)=\ket{\psi}\bra{\psi}=\hat{U}\left(t, t_{0}\right)\ket{\psi(0)}\bra{\psi(0)}\hat{U}^\dagger\left(t, t_{0}\right) =\hat{U}\left(t, t_{0}\right)\hat{\rho}(0)\hat{U}^\dagger\left(t, t_{0}\right)
$$

$ \displaystyle 
\hat{U}(t)=\exp (-i \hat{H} t / \hbar) = \hat{U}(t)=\exp (-i \hat{\mathbb{I}} t / \hbar) \hat{U}(t)=\exp (-i \hat{\sigma}_z t / \hbar) =\exp (-i E_0\hat{\mathbb{I}} t / \hbar)(\cos(\frac{\Delta t}{\hbar})\mathbb{I}+i\hat{\sigma}_{z}\sin(\frac{\Delta t}{\hbar})) = 
 e^{-i\frac{E_0\hat{ \mathbb{I}} t }{ \hbar}} \left(\begin{array}{ll}{\cos{\alpha}} & {i\sin{\alpha}} \\ {i\sin{\alpha}} & {\cos{\alpha}}\end{array}\right) $
\end{solution}
	