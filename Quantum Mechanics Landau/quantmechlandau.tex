\documentclass[a4paper,12pt]{article}
%%% Работа с русским языком
\usepackage{cmap}					% поиск в PDF
\usepackage[T2A]{fontenc}			% кодировка
\usepackage[utf8]{inputenc}			% кодировка исходного текста
\usepackage[english,russian]{babel}	% локализация и переносы
%%
\usepackage{cmap} 
\usepackage{ gensymb }
\usepackage[unicode]{hyperref}
\usepackage{ textcomp }
\usepackage{datetime}
\usepackage{physics}
\usepackage{cancel}
\usepackage{mathtools}
\usepackage[margin=0.7in]{geometry}
\usepackage{fancyhdr}
\pagestyle{fancy}
%%
%%% Дополнительная работа с математикой
\usepackage{amsfonts,amssymb,amsthm,mathtools} % AMS
\usepackage{amsmath}
\usepackage{icomma} % "Умная" запятая: $0,2$ --- число, $0, 2$ --- перечисление

%% Шрифты

%%\usepackage{bbold}
\usepackage{euscript}	 % Шрифт Евклид
\usepackage{mathrsfs} % Красивый матшрифт

\newcommand*{\hm}[1]{#1\nobreak\discretionary{}
	{\hbox{$\mathsurround=0pt #1$}}{}}

\usepackage{graphicx}  % Для вставки рисунков
\graphicspath{{images/}{images2/}}  % папки с картинками
\setlength\fboxsep{3pt} % Отступ рамки \fbox{} от рисунка
\setlength\fboxrule{1pt} % Толщина линий рамки \fbox{}
\usepackage{wrapfig} % Обтекание рисунков и таблиц текстом
\title{Quantum Mechanics Seminars}
\date{\today}
\author{Sergey Barseghyan}
%% Теоремы Определения Леммы
\newtheorem{exercise}{Упражнение}[section]
\newtheorem*{solution}{Решение}
\newtheorem{problem}{Задача}[section]



%% Коммутатор




\begin{document} % конец преамбулы, начало документа
	
	\maketitle
	\section{Семинар 1}
\begin{exercise}
	Покажите, что унитарные матрицы, как и эрмитовы, диагонализуемы. Указание: покажите, что эрмитова и анти-эрмитова
	часть унитарного оператора диагонализуемы совместно
\end{exercise}	

\begin{solution}
	$$
	U^{\dagger} U=U U^{\dagger}=I
	$$
	
	Матрица $A$ диагонализируема если
	$$
	\exists P: \hspace*{5pt} P^{-1} A P=\left(\begin{array}{cccc}{\lambda_{1}} \\ {} & {\lambda_{2}} \\ {} & {} & {\ddots} \\ {} & {} & {} & {\lambda_{n}}\end{array}\right)
	$$
	
	$$
	U = A + B = \underbrace{\frac{U + U^{\dagger}}{2}}_{\text{эрмитова}} \hspace{5pt} +\underbrace{\frac{U - U^{\dagger}}{2}}_{\text{антиэрмитова }}
	$$
	
	Две матрицы $A \text{ и } B$ совместно диагонализируемы $\leftrightarrow$ $\commutator{A}{B} = 0 $
	
	$$
	\begin{array}{cccc}
	\commutator{A}{B}=\frac{1}{4}\left(\left(U + U^{\dagger}\right)\left(U - U^{\dagger}\right) + \left(U - U^{\dagger}\right)\left(U + U^{\dagger}\right)\right)\\
	=\frac{1}{4}\left(U U + U^{\dagger} U - U U^{\dagger} - U^{\dagger} U^{\dagger}\right) - \frac{1}{4}\left(U U - U^{\dagger} U + U U^{\dagger} - U^{\dagger} U^{\dagger}\right) = 0
	\end{array}
	$$
\end{solution}	

\begin{exercise}
	В квантовой механике замена базиса реализуется унитарными преобразованиями $\ket{\psi'} = \hat{U}\ket{\psi}$

	\begin{enumerate}
		\item
		Покажите, что гамильтониан при этом заменяется на $\hat{H'}$ = $\hat{U}$$\hat{H}$$\hat{U}^\dagger$
		\item 
		Последнее утверждение необходимо модифицировать, если унитарное преобразование зависит явно от времени $\hat{U}=\hat{U}(t)$. Покажите, что в таком случае гамильтониан необходимо заменить на $\hat{H'} = \hat{U}\hat{H}\hat{U}^\dagger  - i\hat{U} \partial_t \hat{U}^\dagger$
	\end{enumerate}

	\begin{solution}
		\begin{enumerate}
			\item
		
		$$
		i \hbar \hspace{1pt} \frac{\partial }{\partial t} \ket{\psi} = \hat{H} \ket{\psi}
		$$
		$$
		i \hbar \hspace{1pt} \frac{\partial }{\partial t} \hat{U}^\dagger \ket{\psi'} = \hat{H} \hat{U}^\dagger \ket{\psi'}
		$$
		
		$$
		i \hbar \hspace{1pt} \hat{U} \hat{U}^\dagger \frac{\partial }{\partial t}  \ket{\psi'} = \hat{U} \hat{H} \hat{U}^\dagger \ket{\psi'}
		$$
		
		$$
		i \hbar \hspace{1pt} \frac{\partial }{\partial t}  \ket{\psi'} = \hat{U} \hat{H} \hat{U}^\dagger \ket{\psi'}
		$$
		
		$$
		\hat{H'} = \hat{U} \hat{H} \hat{U}^\dagger
		$$
		\item 
		
		Если же $\hat{U}=\hat{U}(t)$
		
		$$
		i \hbar \frac{\partial}{\partial t} \hat{U}^\dagger(t) \ket{\psi'} =i\hbar\left(\hat{U}^\dagger \partial_t \ket{\psi'} + \partial_t \hat{U}^\dagger \ket{\psi'}\right)
		$$
		
		$$
		i \hbar \hspace{1pt}  \hat{U} \frac{\partial}{\partial t} \hat{U}^\dagger(t) \ket{\psi'} =i\hbar\left(\hat{U}\hat{U}^\dagger \partial_t \ket{\psi'} + \hat{U} \partial_t \hat{U}^\dagger \ket{\psi'}\right)
		$$
		
		$$
		i \hbar \frac{\partial}{\partial t} \ket{\psi'} = \left(\hat{U}\hat{H}\hat{U}^\dagger  - i \hbar\hat{U} \partial_t \hat{U}^\dagger\right) \ket{\psi'}
		$$
		
		$$\hat{H'} = \hat{U}\hat{H}\hat{U}^\dagger  - i \hbar \hat{U} \partial_t \hat{U}^\dagger$$
		
			
	\end{enumerate}
\end{solution}
\end{exercise}


\begin{exercise}
Покажите следующие свойства матриц Паули (по повторяющимся индексам подразумевается суммирование):
 	\begin{enumerate}
		\item 
		Они, совместно с единичной матрицей $\sigma^{0}=\hat{\mathbb{I}}_{2 \times 2}$, представляют собой базис в пространстве эрмитовых матриц $2 \times 2$.
		\item 
		Они удовлетворяют следующими правилами перемножения: $$\hat{\sigma}^{\alpha} \hat{\sigma}^{\beta}=\delta_{\alpha \beta} \hat{\mathbb{I}}+i \epsilon_{\alpha \beta \gamma} \hat{\sigma}^{\gamma} \left( \alpha, \beta, \gamma \in\{x, y, z\}, \epsilon_{\alpha \beta \gamma}  -\text{символ Леви-Чевиты }\right)$$
		\item 
		Они удобно экспоненциируются: $\exp \left(i a n_{\alpha} \hat{\sigma}^{\alpha}\right)=\cos a+i n_{\alpha} \hat{\sigma}^{\alpha} \sin a$ (тут $ \boldsymbol{n} $ — произвольный единичный вектор). Указание: разложите экспоненту в ряд; из-за простого правила произведения матриц Паули, произвольные степени от
		их линейных комбинаций вычисляются достаточно просто
	\end{enumerate}
	
\end{exercise}


\begin{solution}
	\begin{enumerate}
	\item 
	$M \in M_{2}(\mathbb{C})$
	$M=\left(\begin{array}{ll}{z_{11}} & {z_{12}} \\ {z_{21}} & {z_{22}}\end{array}\right)$, где $z_{i j} \in \mathbb{C}$
	
	$$
	\sigma_{0}=\left(\begin{array}{ll}{1} & {0} \\ {0} & {1}\end{array}\right), \sigma_{1}=\left(\begin{array}{cc}{0} & {1} \\ {1} & {0}\end{array}\right), \sigma_{2}=\left(\begin{array}{cc}{0} & {-i} \\ {i} & {0}\end{array}\right), \sigma_{3}=\left(\begin{array}{cc}{1} & {0} \\ {0} & {-1}\end{array}\right)
	$$
	
	$$
	c_{\mu} \in \mathbb{C}
	$$
	Докажем линейную независимость
	$$
	c_{0} \sigma_{0}+c_{1} \sigma_{1}+c_{2} \sigma_{2}+c_{3} \sigma_{3}=\mathbf{o}
	$$
	
	$$
	\left(\begin{array}{cc}{c_{0}+c_{3}} & {c_{1}-i c_{2}} \\ {c_{1}+i c_{2}} & {c_{0}-c_{3}}\end{array}\right)=\left(\begin{array}{ll}{0} & {0} \\ {0} & {0}\end{array}\right)
	$$
	Это система имеет имеет только тривиальное решение
	$$
	c_{0}=c_{1}=c_{1}=c_{3}=0
	$$
	
	Теперь покажем что они покрывают всё пространство $	M_{2}(\mathbb{C}) $
	
	$$
	M=c_{0} I+c_{1} \sigma_{1}+c_{2} \sigma_{2}+c_{3} \sigma_{3}
	$$
	
	$$
	\left(\begin{array}{cc}{c_{0}+c_{3}} & {c_{1}-i c_{2}} \\ {c_{1}+i c_{2}} & {c_{0}-c_{3}}\end{array}\right)=\left(\begin{array}{cc}{z_{11}} & {z_{12}} \\ {z_{21}} & {z_{22}}\end{array}\right)
	$$
	
	$$
	c_{0}+c_{3}=z_{11}, c_{0}-c_{3}=z_{22}, c_{1}-i c_{2}=z_{12}, c_{1}+i c_{2}=z_{21}
	$$
	
	$$
	c_{0}=\frac{1}{2}\left(z_{11}+z_{22}\right), c_{1}=\frac{1}{2}\left(z_{12}+z_{21}\right), c_{2}=\frac{1}{2} i\left(z_{12}-z_{21}\right), c_{3}=\frac{1}{2}\left(z_{11}-z_{22}\right)
	$$
	
	
	\item 
	Проверим
	$$
	\sigma_{x} \sigma_{x}=\left(\begin{array}{ll}{0} & {1} \\ {1} & {0}\end{array}\right)\left(\begin{array}{ll}{0} & {1} \\ {1} & {0}\end{array}\right)=\left(\begin{array}{ll}{1} & {0} \\ {0} & {1}\end{array}\right)=\left(\begin{array}{ll}{1} & {0} \\ {0} & {1}\end{array}\right) \delta_{11}+i \varepsilon_{11\gamma} \hat{\sigma}^{\gamma}
	$$
	
	$$
	\sigma_{x} \sigma_{y}=\left(\begin{array}{ll}{0} & {1} \\ {1} & {0}\end{array}\right)\left(\begin{array}{ll}{0} & {-i} \\ {i} & {0}\end{array}\right)=\left(\begin{array}{ll}{i} & {0} \\ {0} & {-i}\end{array}\right)=\left(\begin{array}{ll}{1} & {0} \\ {0} & {1}\end{array}\right) \delta_{12}+i \varepsilon_{123} \hat{\sigma}_{3}
	$$
	
	$$
	\sigma_{y} \sigma_{z}=\left(\begin{array}{ll}{0} & {-i} \\ {i} & {0}\end{array}\right)\left(\begin{array}{ll}{1} & { 0} \\ {0} & {-1}\end{array}\right)=\left(\begin{array}{ll}{0} & {i} \\ {i} & {0}\end{array}\right)=\left(\begin{array}{ll}{1} & {0} \\ {0} & {1}\end{array}\right) \delta_{23}+i \varepsilon_{231} \hat{\sigma}_{1}
	$$
	
	$$
	\sigma_{z} \sigma_{x}=\left(\begin{array}{ll}{1} & {0} \\ {0} & {-1}\end{array}\right)\left(\begin{array}{ll}{0} & {1} \\ {1} & {0}\end{array}\right)=\left(\begin{array}{ll}{0} & {1} \\ {-1} & {0}\end{array}\right)=\left(\begin{array}{ll}{1} & {0} \\ {0} & {1}\end{array}\right) \delta_{32}+i \varepsilon_{312} \hat{\sigma}_{2}
	$$
	
	

\item 

$$
(\vec{a} \vec{\sigma})(\vec{b} \vec{\sigma})=a_{i} b_{k} \sigma_{i} \sigma_{k}=a_{i} b_{k}\left(\delta_{i k}+i e_{i k l} \sigma_{l}\right)=(\vec{a} \vec{b})+i[\vec{a} \times \vec{b}] \vec{\sigma}
$$

$$
(\vec{a} \vec{\sigma})(\vec{a} \vec{\sigma})=\vec{a} \vec{a}=|\vec{a}|^{2}
$$

$$
e^{X}=\sum_{k=0}^{\infty} \frac{1}{k !} X^{k}
$$

$$
\begin{array}{l}{(\vec{\sigma} \vec{n})^{2}=1} \\ {(\vec{\sigma} \vec{n})^{3}=(\vec{\sigma} \vec{n})^{2}(\vec{\sigma} \vec{n})=(\vec{\sigma} \vec{n})} \\ {(\vec{\sigma} \vec{n})^{4}=(\vec{\sigma} \vec{n})^{2}(\vec{\sigma} \vec{n})^{2}=1}\end{array}
$$

$$
\left\{1+\frac{1}{2 !}\left(a\right)^{2}+\frac{1}{4 !}\left(a\right)^{4}-\ldots\right\}+i(\vec{\sigma} \vec{n})\left\{a-\frac{1}{3 !}\left(a\right)^{3}+\frac{1}{5 !}\left(a\right)^{5}-\ldots\right\}
$$

$$
\exp \left(i a n_{\alpha} \hat{\sigma}^{\alpha}\right)=\cos a+i n_{\alpha} \hat{\sigma}^{\alpha} \sin a
$$

\end{enumerate}

\end{solution}

\newpage
\begin{problem}{\textbf{Осцилляции Раби}}
	На двухуровневую систему накладывается периодическое поле, которое может вызывать переходы между этой парой
	уровней:
	$$
	\hat{H}(t)=\left(\begin{array}{cc}{\varepsilon_{1}} & {V e^{-i \omega t}} \\ {V e^{i \omega t}} & {\varepsilon_{2}}\end{array}\right)
	$$
	В начальный момент времени система находилась в состоянии $|\psi(t=0)\rangle=\ket{\uparrow}$. Определите вероятность обнаружить её
	в состоянии $\ket{\downarrow}$  через произвольное время $t$. Что происходит при резонансе, когда отстройка частоты $\delta \equiv \varepsilon_{1}-\varepsilon_{2}-\omega$ обращается в ноль? Указание: покажите, что от зависимости гамильтониана от времени можно избавиться «переходом во вращающуюся
	систему отсчёта» (rotating wave approximation) — унитарным преобразованием вида $\hat{U}(t)=e^{i \hat{\sigma}_{z} \omega_{0} t}$.
	Чему равна соответствующая частота $\omega_{0}$?
	
\end{problem}

\begin{solution}
	Для начала преобразуем Гамильтониан.
	$$
	\hat{H}^{\prime}=\hat{U} \hat{H} \hat{U}^{\dagger}-i \hbar \hat{U} \partial_{t} \hat{U}^{\dagger}
	$$
	
	$$
	\hat{U}(t)=e^{i \hat{\sigma}_{z} \omega_{0} t} = \cos \omega_{0} t +i \hat{\sigma}_{z} \sin \omega_{0} t
	$$
	
	$$
	\hat{H}^{\prime} = \left( \cos \omega_{0} t +i \hat{\sigma}_{z} \sin \omega_{0} t \right)\left(\begin{array}{cc}{\varepsilon_{1}} & {V e^{-i \omega t}} \\ {V e^{i \omega t}} & {\varepsilon_{2}}\end{array}\right) \left( \cos \omega_{0} t -i \hat{\sigma}_{z} \sin \omega_{0} t \right) + 
	$$
	
	$$ + i\hbar \omega_{0} \left( \cos \omega_{0} t +i \hat{\sigma}_{z} \sin \omega_{0} t \right) \left( \sin \omega_{0} t +i \hat{\sigma}_{z} \cos \omega_{0} t \right)
	$$
	
	$$
	\hat{H}^{\prime} =\left(
	\begin{array}{cc}
	\epsilon_1 - \omega _0 &  V
	e^{i t \omega _0-i t \omega +i t \omega _0} \\
 	V e^{-i t \omega_0 + i t \omega -i t \omega _0} & \omega _0 +\epsilon_2   \\
	\end{array}
	\right)
	$$
	
	$$
	\omega_0 = \frac{\omega}{2}
	$$
	
	
\end{solution}


\newpage

\begin{problem}{\textbf{Два спина}}
	\\
	Найдите уровни энергии и собственные состояния для следующего гамильтониана, описывающего систему двух взаимодействующих спинов 1/2:
	$$
	\hat{H}=-J\left(\boldsymbol{\sigma}_{1} \cdot \boldsymbol{\sigma}_{2}\right)=-J\left(\sigma_{1}^{x} \sigma_{2}^{x}+\sigma_{1}^{y} \sigma_{2}^{y}+\sigma_{1}^{z} \sigma_{2}^{z}\right)
	$$
	
\end{problem}	

\begin{solution}
 Будем решать $$ \hat{H}\ket{\psi}=E\ket{\psi } \qquad \psi \in \mathbb{H}=\mathbb{H}_1 \otimes \mathbb{H}_2 $$
 
 $$ \hat{H}\ket{\psi_1} \ket{\psi_2 }=E\ket{\psi_1} \ket{\psi_2} \qquad \ket{\psi } \in \mathbb{H}=\mathbb{H}_1 \otimes\mathbb{H}_2 $$
 
  
 Выберем базис
 $$
 \ket{\uparrow \uparrow},\ket{\uparrow \downarrow} ,\ket{\downarrow \uparrow} ,\ket{\downarrow \downarrow}
 $$
 
 $$
 \left(\begin{array}{l}\ket{\uparrow \uparrow} \\ \ket{\uparrow \downarrow} \\ \ket{\downarrow \uparrow} \\ \ket{\downarrow \downarrow} \end{array}\right)
 $$
 
 $$
 \hat{H} = \sigma^{x}_1 \otimes \sigma^{x}_{2} + \sigma^{y}_1 \otimes \sigma^{y}_{2} +  \sigma^{z}_1 \otimes \sigma^{z}_{2}
 $$
 
 $$ 
 \hat{H}=-J\left( \left(\begin{array}{llll}{0} & {0} & {0} & {1} \\ {0} & {0} & {1} & {0} \\ {0} & {1} & {0} & {0} \\ {1} & {0} & {0} & {0}\end{array}\right) + \left(\begin{array}{cccc}{0} & {0} & {0} & {-1} \\ {0} & {0} & {1} & {0} \\ {0} & {1} & {0} & {0} \\ {-1} & {0} & {0} & {0}\end{array}\right) + \left(\begin{array}{cccc}{1} & {0} & {0} & {0} \\ {0} & {-1} & {0} & {0} \\ {0} & {0} & {-1} & {0} \\ {0} & {0} & {0} & {1}\end{array}\right)\right)=$$
 
 $$= -J\left(\begin{array}{cccc}{1} & {0} & {0} & {0} \\ {0} & {-1} & {2} & {0} \\ {0} & {2} & {-1} & {0} \\ {0} & {0} & {0} & {1}\end{array}\right)
 $$

Собственные состояния $$
-J\left(\begin{array}{cccc}{0} & {-1} & {1} & {0} \\ {0} & {0} & {0} & {1} \\ {0} & {1} & {1} & {0} \\ {1} & {0} & {0} & {0}\end{array}\right) \left(\begin{array}{l}\ket{\uparrow \uparrow} \\ \ket{\uparrow \downarrow} \\ \ket{\downarrow \uparrow} \\ \ket{\downarrow \downarrow} \end{array}\right)
$$

$$
\ket{1}=\frac{\ket{\uparrow\downarrow}-\ket{\downarrow\uparrow}}{\sqrt{2}} \qquad E = -3
$$

$$
\ket{2}=\ket{\downarrow\downarrow}\qquad E=1
$$

$$
\ket{3}=\frac{\ket{\uparrow\downarrow}+\ket{\downarrow\uparrow}}{\sqrt{2}}\qquad E=1
$$

$$
\ket{4}=\ket{\uparrow\uparrow}\qquad E=1
$$
	
\end{solution}
























	\section{Семинар 2}
\begin{exercise}
	Вычислите среднее значение спина $\langle\boldsymbol{S}\rangle$ и его дисперсию $\left\langle(\boldsymbol{S}-\langle\boldsymbol{S}\rangle)^{2}\right\rangle$ для чистого $\ket{\chi}= (\ket{\uparrow}+\ket{\downarrow}) / \sqrt{2}$ 2 и смешанного $\hat{\rho}=\left(\hat{\mathbb{P}}_{\uparrow}+\hat{\mathbb{P}}_{\downarrow}\right) / 2$  состояний спина $1/ 2$. \textit{Комментарий:}  первая величина — это вектор, а вторая
	— это скаляр, длина вектора. \textit{Указание:} Для частицы со спином $1 / 2$ 2 (например, электрон) \textit{оператор спина} а (собственного
	момента)  равен $\hat{\boldsymbol{S}}=(\hbar / 2) \hat{\boldsymbol{\sigma}}$ (то есть $\hat{S}_{x}=(\hbar / 2) \hat{\sigma}_{x}$ и т. д.).
	
\end{exercise}	

\begin{solution}
	$\langle\boldsymbol{S}\rangle$ = $\bra{\chi}\hat{\boldsymbol{S}}\ket{\chi} = $
	$\displaystyle \frac{\hbar}{2} \cdot \frac{1}{2} \left(\begin{array}{l}{1} \hspace{7pt} {1}\end{array}\right)
	 \left(\begin{array}{c}{\hat{\sigma}_{x}} \\ {\hat{\sigma}_{y}} \\ {\hat{\sigma}_{z}}\end{array}\right) \left(\begin{array}{l}{1} \\ {1}\end{array}\right)= $ 
	 $
	 	\frac{\hbar}{2} \left(\begin{array}{l}{1} \\ {0} \\ {0}\end{array}\right)
	 $
	 $$
	 \left\langle(\boldsymbol{S}-\langle\boldsymbol{S}\rangle)^{2}\right\rangle= \bra{\chi}( 
	 \boldsymbol{S}-\langle\boldsymbol{S}\rangle)^{2}  \ket{\chi} =\frac{\hbar^2}{4\cdot 2} \left(\begin{array}{l}{1} \hspace{7pt} {1}\end{array}\right)
	 \left(\left(\hat{\sigma}_x -1\right)^2+ \hat{\sigma}_{y}^2 +\hat{\sigma}_{z}^2\right)
	 \left(\begin{array}{l}{1} \\ {1}\end{array}\right)
	 $$
	 
	 $$
	 \frac{\hbar^2}{4\cdot 2} \left(\begin{array}{l}{1} \hspace{7pt} {1}\end{array}\right)
	 \left(3\mathbb{I} -2\hat{\sigma}_x +1\right)
	 \left(\begin{array}{l}{1} \\ {1}\end{array}\right) = \frac{\hbar^2}{2}
	 $$
	 
	 
	 
	 $$
	 \hat{\rho}=\frac{1}{2}\left(\begin{array}{ll}{1} & {1} \\ {1} & {1}\end{array}\right)
	 $$
	 
	 $$
	 \operatorname{Tr}\hat{\rho}\hat{\boldsymbol{S}} = \frac{\hbar}{2} \operatorname{Tr}\frac{1}{2}\left(\begin{array}{ll}{1} & {1} \\ {1} & {1}\end{array}\right)
	 \left(\begin{array}{c}{\hat{\sigma}_{x}} \\ {\hat{\sigma}_{y}} \\ {\hat{\sigma}_{z}}\end{array}\right)=\frac{\hbar}{2}
	  \left(\begin{array}{c}\operatorname{Tr}\left(\begin{array}{cc}\frac{1}{2} & \frac{1}{2} \\
	 \frac{1}{2} & \frac{1}{2} \\ \end{array}\right) \\ \operatorname{Tr} \left(\begin{array}{cc}
	\frac{i}{2} & -\frac{i}{2} \\ \frac{i}{2} & -\frac{i}{2} \\ \end{array} \right) \\
	\operatorname{Tr} \left(\begin{array}{cc} \frac{1}{2} & -\frac{1}{2} \\ \frac{1}{2} & -\frac{1}{2} \\	\end{array} \right) \\ \end{array} \right) = \frac{\hbar}{2} \left(\begin{array}{l}{1} \\ {0} \\ {0}\end{array}\right)
	  $$
	  
	  
	  $$
	  \operatorname{Tr} \hat{\rho} \left(\boldsymbol{S}-\langle\boldsymbol{S}\rangle\right)^{2} = \operatorname{Tr} \hat{\rho} \left(\left(\hat{\sigma}_{x}-1\right)^{2}+\hat{\sigma}_{y}^{2}+\hat{\sigma}_{z}^{2}\right) = \operatorname{Tr} \frac{1}{2}\left(\begin{array}{ll}{1} & {1} \\ {1} & {1}\end{array}\right) \left(\begin{array}{cc} 2 & -1 \\ -1 & 2 \\\end{array}\right) = \frac{\hbar^2}{2}
	  $$
	 
	 
\end{solution}


\begin{problem}{Блоховское представление двухуровневой системы}
	\begin{enumerate}
		\item 
		Покажите, что матрицу плотности произвольной двухуровневой системы самого общего вида можно разложить по
		матрицам Паули в следующем виде:
		$$
		\hat{\rho}=\frac{1}{2}(\hat{\mathbb{I}}+\hat{\boldsymbol{\sigma}} \cdot \boldsymbol{n})
		$$
		
		\item 
		При каком условии на $ \boldsymbol{n} $, эта матрица плотности описывает чистое состояние?
		
		\item 
		Вычислите средние значения $\left\langle\hat{\sigma}_{x, y, z}\right\rangle$  по состоянию, описываемому такой матрицей плотности.
	\end{enumerate}
\end{problem}

\begin{solution}
	\begin{enumerate}
		\item
		$$
		\hat{\rho}=\sum_{\alpha} p_{\alpha}\ket{\alpha}\bra{\alpha}
		$$
		
		Так как $ \hat{\rho} = \hat{\rho}^\dagger $  тогда $ \hat{\rho} $ можно раложить по базису 
		$
		\hat{\rho} = \hat{\boldsymbol{\sigma}}\boldsymbol{n} + n_0 \sigma^0
		\qquad
		\operatorname{Tr} \hat{\rho}=1
		$
		
		$
		\operatorname{Tr} \hat{\sigma}^{i}=0 \qquad i \in \left\lbrace 1,2,3\right\rbrace
		$
		
		$
		\operatorname{Tr} \hat{\sigma}^{0}=2 \Longrightarrow n_{0}=\frac{1}{2}
		$
		
		$
		\hat{\rho}=\frac{1}{2}\left(\hat{\hat{\sigma}}^{0}+n^{i} \hat{\sigma}_{i}\right)
		$
		
		$
		\bra{\psi}\hat{\rho}\ket{\psi}\geq 0
		\quad
		\hat{\rho} = 
		\frac{1}{2}\left(\begin{array}{ll}{1+n_{z}} & {n_{x} -i n_{y}} \\ n_{x}+{i n_{y}} & {1-n_{z}}\end{array}\right)
		$
		
		
		$
		\left\lbrace\begin{array}{l}{\left|n_{z}\right| \leq 1} \\ {1-n_{z}^{2}-n_{x}^{2}-n_y^{2} \geq 0}\end{array}\right.
		\Longrightarrow
		\vec{n} \in \bar{B}_{1}(0)
		$
		
		\item 
		
		$
		\hat{\rho}^{2}=\hat{\rho}
		$
		
		$
		\hat{\rho}^{2}=\frac{1}{4}\left(\hat{\sigma}_{0}+n_i \hat{\sigma}_{i}\right)^{2} =\frac{1}{4} \left(
		\left(\overline{n} \overline{\sigma}\right)^{2}+2\left(n_{0} \sigma_0\right)(\overline{n} \overline{\sigma})+\left(n_{0} \sigma_0 \right)^{2}
		\right) = 
		\frac{1}{4} \hat{\sigma}_{0}+\frac{1}{2} n^{i} \hat{\sigma}_{i}+\frac{1}{4}(\vec{n})^{2} \hat{\sigma}_{0} =
		$
		
		$
		=\frac{1}{2}\left(\frac{1+|\vec{n}^2|}{2}\sigma_{0} +n^i \sigma_{i}\right)
		$
		
		$	
		\frac{1+|\vec{n}^2|}{2}=1 \Longrightarrow 	|\vec{n}|^{2}=1 \Longleftrightarrow 
		\vec{n} \in \mathbb{S}^{2}
		$
		
		\item 
		$
		\langle \hat{\sigma_{i}}\rangle = \operatorname{Tr}\left( \frac{1}{2}\left( \hat{\sigma_{0}}+ n^k \hat{\sigma}_k\right) \hat{\sigma}_{i} \right)
		$
		
		
		$
		\operatorname{Tr}\left(\frac{1}{2}\left(\hat{\sigma}_{i}+n^{k}\left(\delta_{ki} \hat{\sigma_{0}}+\varepsilon_{kin} \hat{\sigma}_{n}\right)\right)\right) = 
		n_{i}
		$
		
		$
		\langle\hat{\sigma}\rangle=\vec{n}
		$
		
	\end{enumerate}
\end{solution}	


\begin{problem}{Термодинамика двухуровневой системы}
	Двухуровневая система описывается гамильтонианом $
	\hat{H}=-\boldsymbol{h} \cdot \hat{\boldsymbol{\sigma}}
	$, где $
	\boldsymbol{h}=\left(h_{x}, h_{y}, h_{z}\right)
	$ , и находится при температуре $ T $. Вычислите средние значения $\left\langle\hat{\sigma}_{x, y, z}\right\rangle$.
\end{problem}	

\begin{solution}
	$\displaystyle 
	\hat{\rho}=\frac{1}{Z} e^{-\beta \hat{H}} = \frac{1}{Z} e^{-\frac{1}{T} \hat{H}} = \frac{1}{Z} e^{\frac{1}{T} h_x \hat{\sigma}_{x}+h_y \hat{\sigma}_{z}+h_z \hat{\sigma}_{z}}$
	
	
	$ \displaystyle 
	Z = \operatorname{Tr}	= \frac{1}{Z} e^{\frac{|h|}{T} \left(\frac{h_x \hat{\sigma}_{x}}{|h|}+\frac{h_y \hat{\sigma}_{y}}{|h|}+\frac{h_z \hat{\sigma}_{z}}{|h|}\right)} 
	$
	
	$ \displaystyle 
	\operatorname{Tr}\left(\cos \left(-i \frac{|\vec{h}|}{T}\right)+\sin \left(-\frac{-i |\vec{h}|}{T}\right) \cdot \frac{h_\alpha \hat{\sigma}^{\alpha}}{|\vec{h}|}\right)=
	\operatorname{Tr}\left(\operatorname{ch } \left(\frac{|\vec{h}|}{T}\right)+\operatorname{sh}\left(\frac{|\vec{h}|}{T}\right) \cdot \frac{h_{\alpha} \sigma^{\alpha}}{|\vec{h}|}\right) = 2 \operatorname{ch} \frac{|\vec{h}|}{T}
	$
	
	$ \displaystyle
	\hat{\rho}=\frac{e^{\frac{1}{T}\left(h_{\alpha} \hat{\sigma}^{\alpha}\right)}}{2 \operatorname{ch} \frac{|\vec{h}|}{T}} = \frac{1}{2}\frac{\operatorname{ch}\frac{|\vec{h}|}{T} \cdot \mathbb{I} + \operatorname{sh}\frac{|\vec{h}|}{T} \frac{h_{\alpha} \hat{\sigma}^{\alpha}}{|\vec{h}|}}{\operatorname{ch}\frac{|\vec{h}|}{T}} = \frac{1}{2}\left(\mathbb{I} + \operatorname{th}\frac{|\vec{h}|}{T}\frac{h_{\alpha} \hat{\sigma}^{\alpha}}{|\vec{h}|}\right)
	$
	
	
	$
	\left\langle \hat{\sigma}_{x}\right\rangle =\operatorname{Tr}\left(\hat{\rho} \hat{\sigma}_{x}\right)= \frac{1}{2} \operatorname{Tr}\left(\left(\mathbb{I}+\operatorname{th} \frac{|\vec{h}|}{T} \frac{h_{\alpha} \hat{\sigma}^{\alpha}}{|\vec{h}|}\right) \hat{\sigma}_{x}\right) = \frac{h_x}{h} \operatorname{th}\left(\frac{|\vec{h}|}{T}\right)
	$
	\\
	Аналогично
	\\
	
	$ 	\left\langle \hat{\sigma}_{y}\right\rangle = \frac{h_y}{h} \operatorname{th}\left(\frac{|\vec{h}|}{T}\right) $
	$ 	\left\langle \hat{\sigma}_{z}\right\rangle = \frac{h_z}{h} \operatorname{th}\left(\frac{|\vec{h}|}{T}\right) $
	
\end{solution}	


\begin{problem}
	Рассмотрите двухуровневую систему, описываемую следующим гамильтонианом
	$$\hat{H}=\left(\begin{array}{cc}{E_{0}} & {-\Delta} \\ {-\Delta} & {E_{0}}\end{array}\right)$$
	В начальный момент система приготовлена в состоянии $ \ket{\psi(0)} = \ket{\uparrow} $ . Если бы мы позволилит системе эволюционировать самой по себе, то она совершала бы осцилляции Раби; в частности, через время $T=\frac{\pi \hbar}{2 \Delta}$ мы бы обнаружили её в состоятнии $ \ket{\downarrow} $, c вероятностью $P_{\downarrow}(T)=1$. Однако теперь вместо этого через каждый промежуток времени $\tau \ll T$ мы проводим измерение наблюдаемой $ \hat{\sigma}_{z} $. Определите вероятность $P_{\downarrow}(T)$ в таком случае.
\end{problem}

\begin{solution}
$	\ket{\psi(t)} = e^{\frac{-i E_0 t}{\hbar}} \left(\cos \frac{\Delta t}{\hbar}\cdot\ket{\uparrow}+i \sin \frac{\Delta t}{\hbar}\cdot\ket{\downarrow}\right)$

$$
\hat{\rho}=\left(\begin{array}{c}{\cos ^{2}\left(\frac{\Delta t}{\hbar}\right) \quad-i \sin \left(\frac{\Delta t}{\hbar}\right) \cos \left(\frac{\Delta t}{\hbar}\right)} \\ {i \sin \left(\frac{\Delta t}{\hbar}\right) \cos \left(\frac{\Delta t}{\hbar}\right) \quad \sin ^{2}\left(\frac{\Delta t}{\hbar}\right)}\end{array}\right)
$$
После измерения

$$\hat{\rho} \rightarrow \hat{\rho}^{\prime}=\left(\begin{array}{cc}{\cos ^{2}\left(\frac{\Delta t}{\hbar}\right)} & {0} \\ {0} & {\sin ^{2}\left(\frac{\Delta t}{\hbar}\right)}\end{array}\right)$$

$$
\hat{\rho}(t)=\ket{\psi}\bra{\psi}=\hat{U}\left(t, t_{0}\right)\ket{\psi(0)}\bra{\psi(0)}\hat{U}^\dagger\left(t, t_{0}\right) =\hat{U}\left(t, t_{0}\right)\hat{\rho}(0)\hat{U}^\dagger\left(t, t_{0}\right)
$$

$ \displaystyle 
\hat{U}(t)=\exp (-i \hat{H} t / \hbar) = \hat{U}(t)=\exp (-i \hat{\mathbb{I}} t / \hbar) \hat{U}(t)=\exp (-i \hat{\sigma}_z t / \hbar) =\exp (-i E_0\hat{\mathbb{I}} t / \hbar)(\cos(\frac{\Delta t}{\hbar})\mathbb{I}+i\hat{\sigma}_{z}\sin(\frac{\Delta t}{\hbar})) = 
 e^{-i\frac{E_0\hat{ \mathbb{I}} t }{ \hbar}} \left(\begin{array}{ll}{\cos{\alpha}} & {i\sin{\alpha}} \\ {i\sin{\alpha}} & {\cos{\alpha}}\end{array}\right) $
\end{solution}
	
	
\end{document}