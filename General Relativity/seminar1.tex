
\section{Семинар 1}

\begin{problem}
	Покажите, что прямой мировой линии отвечает именно минимум (а не максимум) действия $S[x]=-m \int_{t_{A}}^{t_{B}} d t \sqrt{1-\boldsymbol{v}^{2}}$,
	то есть максимум собственного времени $s=\int_{A}^{B} d s$ Приведите примеры мировых линий, отвечающих
	наименьшему собственному времени. Чему равно это время?
\end{problem}	

\begin{solution}
	content...
\end{solution}




\begin{problem}
В случае системы нескольких свободных частиц момент импульса равен сумме их моментов:

$$
J^{\mu \nu}=\sum_{s}\left(x_{s}^{\mu} p_{s}^{\nu}-x_{s}^{\nu} p_{s}^{\mu}\right)
$$
Покажите, что сохранение компонент $ J^{0 i} $ 
эквивалентно тому, что центр инерции системы
$$
\boldsymbol{R}=\frac{\sum_{s} E_{s} \boldsymbol{r}_{s}}{\sum_{s} E_{s}}
$$
движется с постоянной скоростью
\end{problem}
\begin{solution}
	$$
	J^{\mu \nu}=\sum_{s}\left(x_{s}^{\mu} p^{\nu}-p_{s}^{\mu} x_{s}^{\nu}\right)
	$$
	
	$$
	J^{0 i}=\sum_{s}\left(t \overline{p}_{s}-E_{s} \overline{r}_{s}\right)=\operatorname{const_1}
	$$
	
	$$
	\sum_{s} E_{s}=\operatorname{const_2}
	$$
	
	$$
	t \underbrace{\frac{\sum_{s} \boldsymbol{p}_{s}}{\sum E_{s}}}_{\boldsymbol{V}}-\underbrace{\frac{\sum_{s} E_{s} \boldsymbol{r_{s}}}{\sum_s E_{s}}}_{\boldsymbol{R}}= \operatorname{const}
	$$
	
	
	$$
	t \boldsymbol{V}-\boldsymbol{R}=\operatorname{const}
	$$
	
\end{solution}
\begin{problem}
	content...
\end{problem}



\begin{problem}
	Выведите уравнение $ m \frac{d u_{\mu}}{d s} = e F_{\mu \nu} u^\nu$
	
\end{problem}	


\begin{solution}
	$$
	S = \int_{t_1}^{t_2}dt  L 
	$$
	
	
	$$
	S = -m \int d t \sqrt{\dot{x}_{\mu} \dot{x}^{\mu}} - e A_{\mu} \dot{x}^{\mu} dt
	$$
	

	$$
	\frac{d}{d t}\frac{\partial L}{\partial \dot{x}^{\mu}} - \frac{\partial L }{\partial x^{\mu}} = 0 
	$$

	$$
	\frac{\partial L}{\partial \dot{x}_{\mu}}= \left(-m \frac{\dot{x}_{\mu}}{\sqrt{\dot{x}_{\sigma} \dot{x}^{\sigma}}} - e A_{\mu}\left[x(t)\right]\right)
	$$

	
	$$
	\frac{d}{d t}\left(-m \frac{\dot{x}_{\mu}}{\sqrt{\dot{x}_{\mu} \dot{x}^{\mu}}} - e A_{\mu}\left[x(t)\right]\right) = -m \frac{d u}{d t} - e \partial_{\nu} A_{\mu}\dot{x}^{\nu}
	$$
	
	$$
	\frac{\partial L }{\partial x^{\mu}} = - e \partial_{\mu} A_{\nu}  \left[x(t)\right] \dot{x}^{\nu}
	$$
	
	$$
	m\frac{d u_{\mu}}{d s} = e \underbrace{\left(\partial_{\mu}A_{\nu} - \partial_{\nu}A_{\mu}\right)}_{F_{\mu \nu}} = e F_{\mu \nu} u^{\nu}
	$$
	
	\begin{problem}
		Рассмотрите частицу во внешнем скалярном поле, которое описывается зависящей от точки
		массой $(x)$ в действии $S\left[x\right] = \int^{B}_{A} \left(- m ds -e A\right)$. Напишите гамильтониан и уравнения движения такой частицы. Покажите, что если $m(x) = m_0+U(x), U(x)\ll m_{0}$, то в нерелятивистском пределе эти уравнения описывают
		частицу во внешнем потенциальном поле $U(x)$
	\end{problem}
	
	\begin{solution}
	$$
	S = \int^{B}_{A} -m(x)ds -e A = \int^{t_B}_{t_A} dt \left(- m \sqrt{1 - v^2} + e\boldsymbol{A}\boldsymbol{v} - e \varphi \right)
	$$
	
	$$
	\boldsymbol{p}_{H} = \frac{\partial L}{\partial \boldsymbol{v}} = \frac{m \boldsymbol{v}}{\sqrt{1-v^2}} + e \boldsymbol{A} = \boldsymbol{p} + e\boldsymbol{A}
	$$
	
	$$
	\frac{m \boldsymbol{v}}{\sqrt{1-v^2}}= \boldsymbol{p}_H-e\boldsymbol{A}
	$$
	
	$$
	\frac{m^2 v^2}{1-v^2}= \left(\boldsymbol{p}_H-e\boldsymbol{A}\right)^2
	$$
	
	$$
	m^2 v^2 =  \left(\boldsymbol{p}_H-e\boldsymbol{A}\right)^2 \left(1-v^2\right)
	$$
	
	$$
	v^2 = \frac{\left(\boldsymbol{p}_H-e\boldsymbol{A}\right)^2 }{\left(m^2 + \left(\boldsymbol{p}_H-e\boldsymbol{A}\right)^2\right)}
	$$
	
	$$
	v = \frac{\left(\boldsymbol{p}_H-e\boldsymbol{A}\right)}{\sqrt{\left(m^2 + \left(\boldsymbol{p}_H-e\boldsymbol{A}\right)^2\right)}}
	$$
	
	
	
	$$	
	1- v^2 = \frac{m^2}{m^2 + \left(\boldsymbol{p}_H-e\boldsymbol{A}\right)^2}
	$$
	
	$$
	H = \boldsymbol{v}\boldsymbol{p}_H - L = \frac{\boldsymbol{p}^2_H - 2 \boldsymbol{p}_H \boldsymbol{A} + m^2  + e^2 A^2 }{\sqrt{\left(m^2 + \left(\boldsymbol{p}_H-e\boldsymbol{A}\right)^2\right)}} + e \varphi
	$$
	
	$$
	H= \sqrt{m^2 + \left(\boldsymbol{p}_H - e \boldsymbol{A}\right)^2} + e \varphi
	$$
	
	$$
	\dot{p}_i = - e \left( \partial_i \phi - \partial_0 A_i \right) - \frac{m \partial_i m}{\sqrt{m^2 + \left(\boldsymbol{p}_H - e \boldsymbol{A}\right)^2}} + v_k e \frac{A_k}{x_k}
	$$
	
	
	$$
	m = m_0 + U , U \ll m_0
	$$
	
	
	$$
	v \ll 1 ,  \boldsymbol{p}_H  - e \boldsymbol{A} = \frac{ m \boldsymbol{v}}{\sqrt{1-v^2}}\approx m_0 \boldsymbol{v}
	$$
	
	
	$$
	\frac{m}{\sqrt{m^2 - \left(\boldsymbol{p}_H - e\boldsymbol{A}\right)^2}}\approx \frac{m_{0}^2}{\sqrt{m_{0}^2\left(1-v^2\right)}}\approx 1
	$$
	
	
	$$
	p_i = m v_i
	$$
	
	$$
	\dot{p}_i = - e \left(\partial_i \varphi  - \partial_0 A_i\right) - \frac{\stkout{m_0} \partial_i U}{\stkout{m_0}} + v_k e \frac{A_k}{x_k}
	$$
	\end{solution}


\end{solution}



	