
\section{Семинар 1}

\begin{problem}
	Покажите, что прямой мировой линии отвечает именно минимум (а не максимум) действия $S[x]=-m \int_{t_{A}}^{t_{B}} d t \sqrt{1-\boldsymbol{v}^{2}}$,
	то есть максимум собственного времени $s=\int_{A}^{B} d s$ Приведите примеры мировых линий, отвечающих
	наименьшему собственному времени. Чему равно это время?
\end{problem}	

\begin{solution}
	content...
\end{solution}




\begin{problem}
В случае системы нескольких свободных частиц момент импульса равен сумме их моментов:

$$
J^{\mu \nu}=\sum_{s}\left(x_{s}^{\mu} p_{s}^{\nu}-x_{s}^{\nu} p_{s}^{\mu}\right)
$$
Покажите, что сохранение компонент $ J^{0 i} $ 
эквивалентно тому, что центр инерции системы
$$
\boldsymbol{R}=\frac{\sum_{s} E_{s} \boldsymbol{r}_{s}}{\sum_{s} E_{s}}
$$
движется с постоянной скоростью
\end{problem}
\begin{solution}
	$$
	J^{\mu \nu}=\sum_{s}\left(x_{s}^{\mu} p^{\nu}-p_{s}^{\mu} x_{s}^{\nu}\right)
	$$
	
	$$
	J^{0 i}=\sum_{s}\left(t \overline{p}_{s}-E_{s} \overline{r}_{s}\right)=\operatorname{const_1}
	$$
	
	$$
	\sum_{s} E_{s}=\operatorname{const_2}
	$$
	
	$$
	t \underbrace{\frac{\sum_{s} \boldsymbol{p}_{s}}{\sum E_{s}}}_{\boldsymbol{V}}-\underbrace{\frac{\sum_{s} E_{s} \boldsymbol{r_{s}}}{\sum_s E_{s}}}_{\boldsymbol{R}}= \operatorname{const}
	$$
	
	
	$$
	t \boldsymbol{V}-\boldsymbol{R}=\operatorname{const}
	$$
	
\end{solution}











	